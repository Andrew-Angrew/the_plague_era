% !TeX encoding = UTF-8
\documentclass[12pt,a4paper]{article}
%\usepackage{russ}

\usepackage[utf8]{inputenc}
\usepackage[english,russian]{babel}
%\usepackage[left=2cm,right=2cm,top=2cm,bottom=3cm]{geometry}

\usepackage{amsmath,amsfonts,amssymb}
\usepackage{amscd}
\usepackage{amsthm}
%\usepackage{geometry}
%\usepackage{fancyhdr}
%\parindent = 0em %ðàññòîÿíèå îò êðàÿ â êðàñíûõ ñòðîêàõ
\usepackage{color}
\usepackage{cmap}

\usepackage{underscore} %нижние подчеркивания вне формул

%для ссылок
\usepackage{xcolor}
\usepackage{hyperref}
\definecolor{linkcolor}{HTML}{799B03} % цвет ссылок
%\definecolor{urlcolor}{HTML}{799B03} % цвет гиперссылок
\hypersetup{pdfstartview=FitH,  linkcolor=linkcolor, urlcolor=blue
	, colorlinks=true}

\usepackage{ulem} %зачеркивание via \sout{}

% чтобы itemize работал без лишних пробелов
\usepackage{enumitem}
\setlist{nolistsep,
	% itemsep=0.3cm,
	parsep=4pt}

\usepackage{color}
\newcommand{\tr}[1]{\textcolor{red}{#1}}

\newcommand{\todo}[1]{\marginpar{\scriptsize \tr{#1}}}

\newcommand{\sep}{
	\begin{center}
		\line(1,0){300}
	\end{center}
}

\begin{document}


\section*{Предисловие}

События, о которых я хочу рассказать, не имеют удовлетворительного описания. Дело в том, что их участники очень сильно отличаются от людей. Трудно даже провести параллели между их органами чувств и нашими. Возможно, было бы правильнее изображать их, как призраков-колдунов, живущих в атмосфере газового гиганта. Или как морских существ с жидкими телами, способными  смешиваться с другими существами и предметами в одном объеме, но при этом четко осознающими и контролирующими границы собственного тела. Это образы лучше описывали бы физическую суть происходящего. Но чтобы передать то, что мне в этой истории кажется ценным, я буду рисовать своих героев людьми, используя в качестве красок узнаваемые образы из человеческой культуры. Иногда этот антропоморфный образ будет ломаться, несмотря на намеренную неподробность описаний. Надеюсь, это вступление позволит сохранить в такие моменты ощущение осмысленности происходящего.

\section*{Встреча}

Одинокий путник пришел из пустыни к окраине города. По нынешним временам он был необычайно стар. Он не был немощным, возраст только придавал достоинство его облику. Только вот уже много поколений никому не удавалось прожить столько лет. Всех рано или поздно убивала чума. Чумой болели все. Каждая жизнь рано или поздно превращалась в изматывающую и безнадежную борьбу с этой болезнью.

Тело путника было по-архаичному мощным и грузным. Такое сложение можно увидеть только у статуй времен золотого века или в самых давних воспоминаниях наследственной памяти. Его движения казались неестественно точными и равномерными, будто подчиненными своеобразной эстетике, имеющей мало общего с природной грацией.

Сквозь кольцо трущоб, путник добрался до центра города -- перенаселенных античных руин. Повсюду в грязи и пренебрежении были разбросаны следы былого великолепия. Роскошные древние дома облеплены нелепыми пристройками, заселенными множеством семей.

Путник проехал мимо шкуры демона, установленной на центральной площади. Шкура была частично невидима для всех чувств, как черно-белая дыра в цветном мире. В месте прикосновения к ней кожа как-будто немела. Другими словами шкура была \textbf{\textit{мертва}}. Более удивительно было то, что она абсолютно не поддавалась всеразлагающему воздействию \textbf{\textit{жизни}}. И потому вместе с другими шкурами была древнейшей в мире вещью, оставшейся еще со времен первого апокалипсиса.

Доехав до центрального храма-дворца путник представился Авиценной, великим странствующим целителем. Он заявил, что умеет лечить чуму, и запросил аудиенции у самого пророка Ходжи\footnote{Ходжа -- наследственный титул потомков исламских миссионеров суфийского течения, предполагающего, что все души - части единого целого.}. Ссылаясь на обычаи родины, Авиценна отказался открыть свой разум и показать искренность своих намерений. Это было откровенно враждебным действием. Но все же небольшая демонстрация \textbf{\textit{силы}} открыла ему двери дворца, для оказания соответствующего приема. Привилегированность могущественных входит во все обычаи. Через некоторое время Ходжа решил принять гостя, несмотря на всю подозрительность его поведения.

Ходжа был не был юн, но его возраст казалось не соответствовал его славе мудреца и уважению которое оказывали ему его последователи. Он выделялся худощавостью и был физически неразвит, а весил наверное чуть ли не вдвое меньше необычайно мускулистого Авиценны.

Начало аудиенции прошло в тишине, -- увидев больных, Авиценна без лишних слов занялся ими. Взрослых чума выматывает постепенно, а к детям приходит периодически в виде острых приступов. Представлены были все возраста и степени тяжести. Казалось, Авиценна решил осмотреть всех от тяжелых к легким. Но когда он добрался до последних пациентов, стоны детей, осмотренных первыми, начали стихать. Все были исцелены. Вот так просто, без заметных манипуляций и инструментов.

-- Я понимаю, что любое предложение высказанное на словах вызывает недоверие. -- сказал Авиценна -- Надеюсь отпечатка моей мысли, хватит в качестве доказательство моей искренности?
С легкой улыбкой Ходжа кивнул.

-- И еще я бы хотел продолжить разговор в максимально узком кругу.

-- Вы же понимаете, что учитывая {\it наши} обычаи, ничего не получится сдержать в секрете.

-- Да, но мне бы хотелось хоть как-то сдержать распространение информации о том, что я собираюсь сообщить. Это помогло бы в деле, которое я намерен предложить.

По знаку Ходжи все, кроме нескольких ближайших советников, покинули зал. Очевидно он был абсолютно уверен в своих силах, особенно в своем доме, и не испытывал страха перед подзрительным и могущественным чужаком. Авиценна погрузился в изготовление оттиска. Мысль постепенно застывала в предоставленном для нее камне, формируя образ столь многогранный и сложный, что подделка его считалась невозможной. При прочтении камня в сознании разом вспыхивали впечатления, которые можно было примерно передать следующими словами:

\begin{quotation}
	\textit{Моя цель -- спасение человечества от гибели. Но этот проект в отличие от моей жизни далек от завершения. Многие детали ради успеха дела я вынужден держать в секрете, поэтому не могу никому открыть свой разум. Я хочу, чтобы Ходжа стал моим приемником. Я уже пытался с более слабыми кандидатами, но они не справились. Поэтому, Ходжа, твое согласие очень важно.}
\end{quotation}

-- Почему вы решили, что именно я вам нужен?

-- Трудно найти равного вам по силе и мастерству. А подвергать сомнению ваши моральные достоинства, особенно в этой стране, кажется кощунством. -- сказал Авиценна с легкой усмешкой -- Также я изучил ваши трактаты по этике и теории распределенных систем и оценил ваш блестящий ум. Все эти качества важны для продолжения моего дела. Уверен подобное сочетание всех добродетелей в одном лице уникально.

В ходе дальнейшей беседы Ходжа понял, что больше информации из Авиценны не вытянуть и вскоре дал согласие. Всего через день он был готов оставить свой пост. Весть об этом еще не успела распространиться на всю общину. Поэтому, как и хотел Авиценна, они вдвоем смогли отбыть инкогнито.

Когда они углубились в пустыню Авиценна сказал:

-- Теперь нам нужно телепортираться. Позволь наложить на тебя заклинание.

Исполнение просьбы поставило бы Ходжу в уязвимое положение. После короткой паузы он спросил:

-- Почему бы вам просто не открыть мне свой разум? Я ведь все равно должен узнать ваши секреты и теперь я изолирован.

-- Горький опыт подсказывает, что прежде, чем это сделать, нужна теоретическая и моральная подготовка. Без телепортации мы не сможем попасть в мой дом, где собраны необходимые инструменты. Кроме того процес обучения предполагает множество ситуаций, когда ты окажешься в моей власти. Прошу, доверься мне.

-- Почему вы предпочитаете логику и ультиматумы взаимопониманию и компромиссам? - с легким вздохом смирился Ходжа.
Оставив вопрос без ответа Авиценна сотворил заклинание.



\section*{Первый апокалипсис}

Огненные линии в небе завершились далеким грохотом. Никто не смог истолковать этот знак, и мы отправились по растаявшему в воздухе следу. На самой границе наших кочевий мы нашли место падения. Глубокая рана в теле земли уже затягивалась: мертвый пепел превращался в почву, которая по краям кратера успела дать побеги. Но несмотря на старания жизни вернуть свое, никто не мог припомнить такого количества мертвого вещества, на сколько бы поколений вглубь памяти не удавалось заглянуть. Наблюдать мертвое было странно и интересно. Даже сосредоточившись можно ощутить лишь очень небогатую гамму ощущений. Как будто трогаешь что-то обмороженными руками. Привычные манипуляции с ним не удаются. Как будто толкаешь что-нибудь, но внезапно не встречаешь сопротивления.

Вещество в центре кратера становилось все более прозрачным и доступным восприятию по мере того, как жизнь просачивалась внутрь. И вот мы заметили в нем смутное движение гигантских, черезвычайно плотных фигур. Их вид казался не просто странным. Он был невозможным. Одни чувства вместе с землей содрогались от мощи их движений. Другие видели на их месте только идеальную, пугающую пустоту. Все что приближалось к ним тоже постепенно начинало ускользать от наших чувств. Попытки прощупать их вызвали длинные струи ярчайшего пламени. Но никого из нас не задело -- отпор гигантов был разрушительным, но неуклюжим. Издалека мы попробовали спровоцировать их еще и вскоре пришли к выводу, что они нас не видят. Они не замечали никакой активности вокруг них, пока та не выливалась в грубое физическое воздействие. Но как только рядом что-то перемещались, они испепеляли это, игнорируя даже самые очевидные следы, ведущие к источнику соответствующих манипуляций.

Мертвые существа не казались особенно опасными, но производили впечатление чего-то пугающе неправильного. Уходя на стоянку, мы решили, что это дурной знак.

Когда мы в следующий раз пришли проведать кратер, мы ожидали что все мертвое вещество будет уже поглощено жизнью. Но его стало только больше. Потом мы увидели самое страшное. Вещество кратера ощущалось как мертвое, но внутри кипела деятельность, как будто оно было живым. Мы заметили, как смерть борется с окружающей его жизнью и постепенно разъедает ее, как кислота. Мы долго пытались что-нибудь с этим сделать. Но раз за разом получали отпор и в конце концов привлекли внимание демонов (так мы окрестили мертвых гигантов). Больше всего пугало то, что все мертвое реагировало на нас как единое целое. Как одно невозможно большое и могучее существо, способное легко раздавить всех нас. Как и демоны оно не отличалось умом, но он был ему и не нужен, чтобы оттеснять нас.

Вскоре нам пришлось бежать с наших земель.

\sep

Мою семью миновали лишения бесконечного похода. Но все равно мое счастье и детство кончились рано. Сначала меня стала преследовать беспричинная тревога. Желая понять ее причину, я прислушался к своим чувствам. И тогда из глубин сознания поднялся холодный неумолимый ужас... Дальше я помню только, как уже очутился в объятиях у родителей, они молча успокаивали меня. А потом я начал вспоминать... чуждое, непонятное, мертвое зло, что пожрало каждый клочок земли, пройденной за долгие годы... изнуряющее бегство от края черной пропасти размером в пол-мира... постоянное расчеловечивающее ожесточение борьбы за жизнь с другими беженцами...

Ко дню инициации я помнил уже достаточно, чтобы четко осознать, насколько случайно получилось так, что я живу. Что я нисколько не лучше многих тех, кто сгинул в хаосе похода вместо моих предков. Опыт многих поколений говорил мне, что я вероятно рано умру, чтобы мои близкие могли умереть чуть позже.

Обреченность довлела над всеми. Кто-то добровольно уходил из жизни. Кто-то с мрачным упорством ходил на край мертвого моря, пытаясь как-то на него повлиять. И однажды случилось невероятное. Одна маленькая, почти не ощутимая частичка мертвого вещества откликнулась на манипуляцию. С помощью нее удалось захватить еще и еще. И вот в распоряжении героя оказался уже приличный объем. Секрет было трудно объяснить -- только долгая ментальная синхронизация помогала. Несмотря на это, он быстро разлетелся по племенам беженцев. В течение десятка лет мы достигли мастерства, позволявшего мечтать о том, тобы владеть мертвым морем, как мы когда-то владели всем живым миром.

Но мертвое море продолжало наступать. Наконец, мы уже достаточно овладели мастерством демонической жизни, чтобы обмануть море и перемещаться в нем. Некоторые плмена отважились снарядить экспедиции вглубь моря. Когда мы полностью погрузились в море демоническое мастерство стало работать гораздо эффективнее. Как будто все вещи стали очень легкими и прозрачными, а силы трения были отменены. Демоническое мастерство всегда казалось далеко отстающим аналогом привычного мастерства манипуляции жизнью. Но внутри моря стали открываться возможности немыслимые для мастерства жизни.

Однажды мы решили, что можем сразиться с морем его же силами. Мы прокрались в черную глубину так далеко, как только смогли, чтобы демонические способности стали максимально эффективны. Наконец, самый способный из нас осуществил задуманное. Он как бы закричал. 
Но этот крик вместо того, чтобы затухать только набирал силу в среде мертвого моря. Волна распространяясь все дальше набирала колоссальную мощь. Потом мы ощутили, как где-то на расстоянии, недоступном для обычного восприятия, обрушились горы...

В ответ на нашу наглость мертвое море вокруг нас умерло, лишилось своей демонической жизни. Превратилось в обычную мертвую непрозрачную массу. Было очень больно. Как оказалось мы бессознательно включили в наши тела частички демонической материи. А теперь они внутри нас обратились пепел. Чем лучше владение демоническим мастерством, тем больше был урон. Некоторые из нас умерли. Мы были ранены и пойманы в мертвом веществе, как в зыбучем песке. Поскольку я пострадал меньше всех, было решено телепортировать меня домой просить подмогу. Но как только я оказался дома, гиганский взрыв поглотил место нашей вылазки вместе с огромным умершим куском моря.

Из нашей группы выжил только я, но это была победа! Началось отвоевание земли. Море по частям отмирало, пытаясь защититься от наших атак. Затем кому-то удалось отравить его и убить полностью. Остатки демонов еще долго бродили по оживающему миру, постепенно погибая от рук все новых героев.

\section*{Ученичество}

Авиценна и Ходжа оказались в зале, заполненном одинаковыми безликими статуями. Вокруг них валялись осколки, оставшиеся по-видимому от таких же статуй, которые были использованы, чтобы собрать здесь их новые тела. Новое тело ощущалось до странности легким.

-- Как ты думаешь, где мы? -- спросил Авиценна.

Ходжа задумался и предположил:

-- На Луне?

Широкая улыбка подтвердила его догадку.

-- Теперь, когда мы у меня дома, я бы предпочел, чтобы ты звал меня Полимегист. Это мое настоящее имя.

-- Хорошо, -- глаза Ходжи немного расширились выдав, что он узнал имя одного из величайших магистров времен перед вторым апокалипсисом.

-- Пойдем, покажу тебе все.

Дом Полимегиста оказался большим дворцом времен золотого века. С огромными помещениями и множеством изящных излишеств, на которые в эпоху Чумы было не принято тратить время. На пути им встречались немногочисленные жители -- по-архаичному избыточно мускулистые люди, одетые в старинном стиле. Наконец они пришли в просторную многоэтажную библиотеку.

-- Здесь тебе предстоит провести много времени. -- сказал Полимегист, -- В первую очередь я хочу, чтобы ты изучил теорию эволюции. А более конкретно разобрался с ответом на главный вопрос: ''Почему от поколения к поколению у человечества растет уровень мастерства?''. Что ты сейчас об этом думаешь?

-- В воспоминаниях предков можно увидеть как это происходит. В течении жизни люди оттачивают свои навыки. Потом часть накопленного умения передается их детям, которые начинают свое развите из лучшей стартовой точки. Таким образом, прогресс идет благодаря усилиям, которые люди вкладывают в развитие своего мастерства. К тому же у лучших мастеров, как правило, больше детей. Это ускоряет процесс.

-- Есть и другая возможность. Что если наследственность накладывает ограничения на возможности человека? Под наследственностью я понимаю совокупность свойств организма, которые передаются от родителя к детям, но не могут изменяться в течение жизни. Тогда на первый взгляд, популяция будет вечно скована наследственностью ее первых особей. Но что если передача наследственности происходит неточно? И ребенок может случайным образом получить мутацию, то есть наследуемое свойство, которого не было у родителей. Мутации могут влиять на мастерство как положительно так и отрицательно. Большинство из них вообще никак не влияет. Но те из них что положительно сказываются на мастерстве, будут давать своим носителям преимущества при размножении. Полезные мутации однажды появившись будут постепенно захватывать популяцию, а вредные с высокой вероятностью будут исчезать.

-- На мой взгляд твое объяснение более сложное и привлекает дополнительные гипотезы, вроде существования врожденного предела развития способностей. Так что я предпочитаю считать по-своему. По крайней мере до тех пор пока не встречу явление, необъяснимое в рамках моей теории.

-- Могу привести такой пример: возьмем много пар сиблингов между рождениями которых прошло скажем 10 лет. Если измерять их уровень мастерства в одинаковом возрасте, окажется что поздние дети гораздо сильнее. Но если подождать, пока оба из них достигнут пика своих сил, заметной разницы не будет. --- Полимегист сделал паузу. --- Правда с этим доводом есть несколько проблем.

--- Например, по моему опыту, мастерство продолжает расти всю жизнь хоть и с замедлением. Так что этот предполагаемый пик могущества видимо настолько далек, что люди просто до него не доживают.

--- По крайней мере можно увидеть как ранние дети постепенно догоняют поздних. А ещё можно обратиться к материалам исследований времен золотого века, хранящимся здесь. Более фундаментальная проблема состоит в том, что уровень мастерства -- крайне размытое понятие и для любых его измерений приходится использовать прокси-метрики\footnote{прокси-метрика -- формальный показатель разработанный, чтобы аппроксимировать нечетко определенную величину. Например, IQ -- прокси-метрика ''общего уровня интеллекта''.}. Поэтому одного -- даже самого качественного -- исследования не хватит, чтобы однозначно разрешить наш спор. А чтобы добиться надежного понимания, нужно ещё исключить множество альтернативных гипотез. Поэтому на эту тему было проведено огромное число исследований. Я хочу, чтобы ты ознакомился с основными работами и своим умом опробовал их выводы на прочность. Многие из них изучают другие виды или наследование совсем других признаков. Однако все вместе они образуют цельную картину того, как может протекать эволюция вообще.


% Многие исследования изучают особенности распределения мастерства по генеалогической схеме популяции. При анализе много интересной математики вылезает -- начал рассказывать Полимегист с оживлением, явно сдерживаясь от дальнейших подробностей.
% колоссальный труд и информационные технологии золотого века.



\sep

--- Привет! --- рядом со столом Ходжи оказалась девушка, --- Я пришла проведать, как у тебя дела, и поболтать. Я Асклепия\footnote{Асклепий --- бог врачевания в древнегреческой мифологии} --- сказала она энергично протянув руку. Внешность её была вполне взрослой, но из-за мимики и интонаций она производила впечатление почти девчонки.

После некоторого промедления Ходжа вспомнил, что в соответствии с древними обычаями, руку полагается пожать. Тем временем было даже немножко смешно видеть, как Асклепия с энтузиазмом продолжает тянуть к нему распахнутую ладонь.

--- Читаю, к чему можно прийти, если долго считать горошины\footnote{\tr{Тут будет ссылка про законы Менделя.}}, --- Ходжа встал вложил руку в предложенную ладонь. --- Очень приятно. Особенно учитывая, что со мной толком никто не разговаривает.

--- Полимегист запретил рассказывать тебе о ряде вещей. Все боятся разболтать тебе что-то лишнее.

--- А ты не боишься?

--- Думаю в будущем, нам предстоит много совместной работы. Мне не терпится познакомиться. Для начала я бы послушала, как магистр теологии ответит на один вопрос. \todo{фентезийная, но узнаваемая версия дилеммы вагонетки} Представь, что ты видишь, как вагонетка едет на пятерых людей привязанных к рельсам. Ты можешь перенаправить её на другой путь, но там тоже привязан один человек. Что будешь делать?

--- Если вагонетка умозрительная и может подождать я бы ментально синхронизировался со всеми шестерыми до достижения консенсуса по данному вопросу, а затем действовал бы в соответствии с ним.

--- А если не может?

--- Я бы действовал в соответствии с моими предположениями о потенциальном консенсусе.

% про приняте ответственности за убийство одного

--- Я ожидала более глубокого вопроса на такой философский вопрос.

--- Для меня это очень практический вопрос. Видимо, это отличие --- часть широкой пропасти между нами. Я всю жизнь прожил в \textbf{\textit{единодушии}}, чувствуя желания и страдания всех членов общины. Там от принятия подобных решений нельзя спрятаться. И кажется нелепым перекладывать ответственность за них на обычаи и законы.

Внезапно её выражение лица из улыбающегося стало сосредоточенным. Взгляд начал медленно отсутствующе блуждать по залу. Как будто она напряженно прислушивалась к чему-то внутри себя.

--- Извини, не обращай внимания. --- сказала она очнувшись, --- Можешь со мной говорить, пока я концентрируюсь на своих божественных обязанностях. Я все осознаю в это время.

--- Значит ты та самая Асклепия, в честь которой зовут остальных.

--- Ну да, --- тихо сказала она, опустив взгляд на носок своей ноги.

--- Из богов древности ты одна отвечаешь на молитвы. Какая удача с тобой поговорить. Я столько хочу знать про богов.

--- Как ни странно, Полимегист может рассказать гораздо больше моего. Что-то мне подсказывает, ты больше хочешь понять, кто такие боги, а не каково ими быть.

--- Тем не менее, это тоже интересно. -- сказал Ходжа.

Их беседа затянулась до глубокой ночи (хоть разделение на день и ночь было чисто формальным здесь).

% интро про богов
% про закостенение личности

\sep


\tr{\Large \center Дальше пока не читаемо}
\newpage

--- Расскажи мне про богов, --- Сказал Ходжа после очередной беседы о генетике, --- До встречи с тобой я потратил много усилий, чтобы разузнать о них. Но мне удалось собрать лишь крохи более или менее достоверной информации. Похоже на Земле это знание утрачено.

--- Во-первых, боги не являются сверхъестественными сущностями, наделенными совершенно особой силой, как принято рассказывать о них на Земле сейчас. Их создание -- одно из высших достижений человеческого искусства, направляемого наукой. Бог это сложный \textbf{\textit{артефакт}}\footnote{Артефакт -- заклятый предмет, защищенный от стремительного разложения жизнью подобно живым существам и как правило наделенный полезными свойствами.}, тесно срощенный с мозгом своего аватара и позволяющий ему творить заклинания запредельной сложности.

--- А есть ли у бога самостоятельное сознание в каком-то виде?

--- Нет. Грубо говоря, это такой же протез, как искусственная рука, только возможности у него намного больше. Сам процесс его создания довольно похож на постепенное выращивание и овладение тканями новой руки, только гораздо сложнее.

--- Разве для исполнения заклинаний бог не должен принимать одновременно множество мелких решений? Скорее это похоже на гигантское разрастание мозга, чем на лишнюю руку.

--- Пожалуй ты прав, хотя руку тоже не стоит недооценивать в плане количества неосознанных решений, необходимых для ее функционирования. Но все же причин видеть личности в богах не больше чем в игрушечных големах. И те и другие действуют по вполне определенным механическим правилам, их поведение легко предсказать, зная начальные условия.

--- Но если так рассуждать то существо обладающее сознанием не может описываться законами физики. Я думал ты материалист.

--- Я имел ввиду, что поведение бога легко предсказуемо мной. Я знаю, как он работает и любой плод его интеллектуальной деятельности впринципе я мог бы получить сам. Поэтому у меня как-то язык не поворачивается назвать его личностью. К тому же по моему опыту, бог -- это такая вещь, которую я могу разобрать и собрать по-новому как захочу. То есть я действительно понимаю что это такое. И как мне кажется, сознание в этой конструкции не заложено.\todo{слишком надменно?}

--- Звучит как оправдание любого мыслимого насилия над игрушечными големами. Страшно подумать, что было бы будь у нас создатель и рассуждай он так же.


\section*{Конец золотого века}

Выдающееся мастерство во все времена возвышало своих обладателей над другими. Но со временем стало заметно, как выдающиеся таланты возвышают над самим собой и все человечество. 
Научившись вкладывать самые разные умения в артефакты мы отделили их от единоличных прежде хозяев. Теперь многие смогли воспользоваться возможностями немногих. Когда же были отточены технологии массового производства артефактов, человечество почти избавились от необхдимости работать ради выживания. Начался век просвещения, рсцвета искуств, предельного уважения к человеческому потенциалу и индивидуальности. 

Материальные отношения перевернулись с ног на голову. Обладатель уникального умения, которое удалось растиражировать через артефакты теперь мог стать производителем огромного количества благ. Теперь элита~---~это не те, кто берет, а те, кто производит. Средневековое угнетение сменилось лизоблюдством. Множество враждующих деспотий сменилось единым мировым обществом, почти анархическим, скреплённым лишь тонкими нитями престижа.

Больно смотреть глазами беззаботного предка на этот преисполненный светлыми надеждами мир. Как наивно было думать, что это и есть должное положение вещей. Что мы могучие хозяева своей судьбы. Лишь мудрейшие из нас страшились конца, готовясь ко второму апокалипсису. Большинство не считало его угрозой бесконечно возросшим силам человечесва. Ведь нас стало так много, а мастерство каждого из нас так выросло по сравнению с мифической древностью первого апокалипсиса.

И конечно Апокалипсис пришел снова. Но совсем не так как раньше. Вместо моря демонической жизни, с неба обрушилось море мгновенно испепеляющего огня. И демоны. Но не такие, как предже. Гораздо сильнее и умнее. Только сильнейшие из нас могли хоть как то с ними бороться.


О битве почти ничего не известно. 

Но не это было концом.

Сначала чума забрала старость, потом детство.

Чума не похожа, на остальные болезни. Она со временем становится все летальнее и тяжелее, нет так как ведут себя обычные болезни (? сноска с расшифровкой этого соображения и дисклэймер, что этот тезис ставится под сомнение современными биологами). Кроме того она меняется везде одинаково, в то время как от обычной болезни при таком темпе мутирования следовало бы ожидать возникновения разных штаммов. Такое ощущение, что вся чума в мире -- едины организм как мертвое море. (?Есть в ней что-то демоническое.)

Всегда может быть хуже.

\section*{Наречение}
Ключевая роль индивидуальных талантов для прогресса

\sep

Беседа про религию. Мысль, что можно смотреть на учение о боге разделенном на души людей абсолютно материалистично: совокупность людей не является личностью, однако со временем эта совокупность может организоваться в систему, являющуюся таковой.

Мистицизм придает этой теории... силу
так это хитрость?
Ты же понимаешь, что у нас не может быть хитростей. Просто я хотел показать, что с точки зрения практических следствий мое метафизическое учение эквивалентно абсолютно материалистичной доктрине. Значит мистическая версия столь же рациональна, как и материалистичная. Я предпочитаю мистическую, поскольку она тесно сплавлена с моей этической системой.

Мистические постулаты (кажущиеся смелыми утверждения про мир) превращаются в тавтологии при правильной подстановке понятий. Поэтому с практической точки зрения моя метафизика про мир ничего не говорит. То что из нее выводится -- это этическая система, которая логична насколько я смею судить. Поэтому мое мировоззрение нельзя обвинить в нерациональности т. к. выбор базовых ценностей, скажем так, дело вкуса.

Моя теория непротиворечива с т. к. имеет материалистическую модель
\sep

Планета демонов

\sep

наречение Ариста

\section*{Победа единодушия}

способность достигать консенсуса -- не само собой получившееся свойство, а результат отбора и воспитания.

Возникновение нового чувства -- ощущение степени синхронизации с единодушием. В случае, если синхронизации нет возникает острое желание ее достичь.

\section*{Испытание}

Ты хороший человек. Я понимаю, что ты не хочешь меня убивать, но вынужден. Наша встреча все же была к лучшему.

Арист высказывает мысль про контакт с демонами


\section*{История Полимегиста}

Молодость, обретение имени. 
Полимегист в качестве испытания на степень магистра в области боевой магии берется за освобождение одного из последних царств от тирании могущественного мага, играющего в божество.
Кроме тщеславия и романтического настроя на подвиги и приключения Полимегиста сподвиг на эту авантюру такой вопрос: Мне кажется, что я хороший человек. Но почему я так думаю? Я не испытываю ни к кому зла, веду себя благородно, чуствую что поступаю по жизни правильно. Но вдруг только потому, что у меня все есть и благородтво мне ничего не стоит? Нужно испытать себя, подвергнуть риску ради благой цели, чтобы узнать правда ли я тот, кем себя считаю.

\tr{придумать имя тирана} какое-нибудь фараонское имя. Хор-???

\tr{придумать имя агента}
Арат -- отсылка к Арату Сикионскому.

Арат - родственник "фараона" (пра-пра-правнук его деда). Часть секретов фараонской династии содержится в его наследственной памяти. Начальник дворцовой охраны (формально, власть его сильно ограничена ангелами, по факту он начальник полицейских структур, укомплектованных обычными людьми). Искусный (ментальный) лицемер, двоемыслец и самохирург -- умение полученное по наследству и развитое им, чтобы при сканировании разума уметь казаться абсолютно лояльным.


Попросил у фараона аудиенции под предлогом того, что раскрыл заговор.



---

знакомство с Аглаей:
Алгая познакомилась с Полимегистом, когда он временно преподавал ей на чем-то вроде летней школы)
Он это делал с целью отдохнуть от своих многочисленных проектов (нужно развернуть). Они сблизились т. к. он старался тусить со студентами. Полмегист заметил что все больше времени проводит с Аглаей, потому что его сильно притягивает ее немного детское наивное и искреннее восхищение им. Он удивился тому как сильно ему хотелось подобного внимания. Когда они впервые поцеловались он понял, что она -- потомок Агента и помнит его героическое прошлое, а также восхищается тем, какой он интересны преподаватель, собеседник и мечтатель.
Но семестр кончился и они разъехались по своим делам.
Впоследствии они поддерживали контакт но не отношения.

Отсутствие у Полимегиста умения поддерживать отношения вне социально одобренных сценариев. Грубо говоря он мог завязать общение на тусовке, где предполагается, что люди будут общаться. Но самому взять на себя инициативу и пригласить кого-то встретиться -- требовало от него решимости. Ему было комфортно общать в присутствии понятных норм того, что можно, а что нельзя. А когда нужно совершать социальные поступки на свой страх и риск -- он слишком сильно боялся. Наверно можно описать это как склонность к "катастрофизации" в области общения: когда нужно совершить действие на ум приходят самые негативные последствия и он либо много времени тратит на то чтобы выработать линию поведения защищенную от множества этих надуманных угроз, либо пасует. Когда есть социальная норма, дающая ему право действовать как он хочет, для него это облегчение (отсутствие таких норм ведет к ступору).

Кроме того он стеснялся того, что не любит ее так, как она его (точнее ему было стыдно). Имею ли я право на ее любовь? Предложить встречаться в подобных обстоятельствах казалось ему "нечестным", эгоистичным предложением. С дргой стороны напрямую оттолкнуть ее он не решался.

\section*{История Аглаи}
Любовь превратилась для Аглаи в "священное" чувство, которому она подчинила все прочие ее интересы. Она поддерживала контакты (через переписку) с Полимегистом, хотя романтические темы поднимать не смела.
Она хотела найти повод для работы с Полимегистом, чтобы можно было быть вместе. Но вскором времени поняла, что как маг она сильно не дотягивает до его уровня и не может работать с ним на равных.

Тогда она решилась стать новым аватаром бога Асклепия (старый аватар некоторое время назад почил). Сначала она не подошла по совместимости своего сознания с богом (как и множество других кандидатов). Тогда она пошла на то, что обеспечило бы ей всеобщее презрение, стань это известно. Она раздобыла запрещенную литературу по ментальной хирургии. Собрала множество отпечатков мысли оригинального Асклепия, чтобы в чувственных оттенках и дополнительных смыслах его мыслей ухватить суть его индивидуальности. К счастью Асклепий считал, что у него много глубоких осознаний, достойных изготовления оттиска. Добилась доступа к богу, чтобы получать обратную связь о том, как те или иные изменения в ее разуме влияют на совместимость. Также она взяла свои воспоминания о романе с Полимегистом, запечатлела свои ощущения в слепках, чтобы следить за тем, чтобы в процессе хирургии не потерять способность испытывать те же эмоциональные реакции на те события (смысл в том чтобы сохранить свойства личности определющие текущее отношение Полимегиста к ней). Затем она начала изготавливать монументально большие слепки соединяющие в себе ощущения из своих воспоминаний и записей Асклепия. Тут пригодилась ее способность к эмпатии -- нужно было представлять себя кем-то кто испытывает определенное сочетание эмоций. Некоторые ощущения трудно было совместить в один образ -- приходилось искать компромиссы, во вторых сами большие образы нужно было сделать непротиворечивыми межд собой. Затем большие образы использовались для направления ментальной хирургии.

Нужная ментальная хирургия была столь объемной, что провести ее сознательно не представлялось реальным, то есть нужно было использовать заклинания для "автоматизации" множества мелких изменений. А для этого нужна была немалая смелость одно дело просто продумать и составить сложное заклининание, другое дело -- применить к собственному мозгу, ведь расплата за баги может быть сколь угодно суровой. И через некоторое время ей удалось добиться уровня совместимости, достаточного чтобы вступить во владние богом и начать перестраивать его уже под себя. В процессе работы Аглая многим в себе пожертвовала, но за сохранением некоторых вещей внимательно следила. И главными из этих вещей были ее любовь к Полимегисту и те черты, которые как она чувствовала нравились в ней Полимегисту.

Деталь: куратор проекта бога Асклепия наверняка заподозрил, Аглаю, но почел за лучшее промолчать т. к. функционирование этого бога -- гарант долголетия и здоровья магистров, и проволочки с поиском нового аватара напрягали верхушку как ничто другое. Куратор чувствовал вину за то, что подыграл Аглае. У Куратора была еще и личная заинтересованность т. к. он был стар и нуждался в поддержке бога медицины.

Вернувшись домой Аглая видит гигантский ворох сложнейших слепков мысли, -- побочный продукт полугодовой работы. Она подумала, что вышла довольно далеко за пределы прочитанных учебников и ее проект по самохирургии -- работа высокого уровня. Жалко что придется все уничтожить -- как спрятать всю эту кучу она не представляла. 

Аглая скрыла за высокомерием то, как сильно она изменилась. Но работа с Полимегистом заладилась и они вновь стали близки, хотя она вела себя несколько отстраненно. И лишь годы спустя, когда она уверилась в постепенно растущей ответной любви Полимегиста, они поцеловались вновь и он узнал, на что она пошла ради него. Он был поражен силой ее любви, которая за это время только выросла. Еще он чувствовал вину за то, что его нерешительность заставляла Аглаю страдать от мук неразделенной любви и вынудила к отречению от части собственной личности. Но каким-то образом связь их только окрепла.

Аглая чувствовала на себе груз божественной ответственности и не могла избежать сравнений со своим предшественником. Но временем Асклепия по общему мнению превзошла его. Пусть тому легче давались навыки медицинских манипуляций, Асклепия компенсировала это упорным трудом и систематическими упражнениями. Асклепий также изящнее мог умел разрешать встающие на пути его исследований загадки природы. Но в отличие от него (срытного и ревностно хранящего секреты) Аглая много работала в коллаборациях. умение слушать, перекрестное опыление идеями, отсутствие собственнического отношения к идеям (и игнорирования чужих идей), грамотное использование экспертизы коллег. Все это делало ее лучшим ученым, чем Асклепий. Много лет спустя, когда накопится статистика по заменам аватаров богов станет ясно, насколько это нетипичная ситуация (когда приемник превосходит предшественника).

\section*{Битва с демонами}
Повествование из памяти Полимегиста, но главны герой -- Александр, магистр-рыцарь. Хочется показать, что победа над демонами -- его заслуга.

Сам александр -- глава военного департамента магистериума. Департамент периодически хотят упразднить за ненадобностью в золотой век, а также из-за угрозы узурпации власти. Однако этого не происходит из-за личного влияния Александра, уважения которым он пользуется и его авторитета, как талантливого администратора. Александр ненамного старше Полимегиста, но последний почему-то всегда относился к нему как к старшему из-за его спокойной харизмы.

Заслуги Александра:
подготовка
	продавливал модификацию проекта энергетических станций в систему орбитальной обороны
	убежища в шахтах
	держал в готовности инфраструктуру эвакуации населения
	держал свое ведомство в готовности
	держал на готове запретное заклинание телепатического чата
руководство
	настоял на том, чтобы нападать синхронно с возможностью провести обстрел с Луны
	транслировал уверенность по чату, в данной отчаянной ситуации она почти гипнотизирующе действовала на своевольных обычно магистров. Благодаря только этому многие рискованные приказы оказывались выполненными.
	грамотное и хладнокровное руководство
	
	



\section*{Переворот Полимегиста}
разговор и схватка на совете

Александр тяготился опасениями, что после своей победы и смерти большинства магистров, которые могли бы ему противостоять, он установит диктатуру. 

совет
Александр: нужно как можно скорее восстановить обескровленные институты власти, вернуть им прежнюю стабильность (а Полимегист мешал этому и оттягивал совет, в серете лихорадочно подгребая под себя полномочия в автоматизированных системах и хакая богов).

Деталь: Полимегист понимал, что выглядит как умник-выпендрежник с непомерным чсв, размахивающий странной теорией. Да еще предлагающий на ее основании проводить геноцид.

Полимегист излагает свой план.
тезисы Полимегиста:

1. основа процветания -- уникальные таланты, в основе всех технологий тоже лежат они. Остальное в технологиях -- дело техническое, можно научить почти любого (т. е. все явное знание -- это легко). И много кто может развивать эту часть. Короче по интеллекту тоже есть ограничения, но оно не является узким местом. Все согласны. Note: надо классно обыграть этот момент. Ведь для читателя очевидно, что наша цивилизация основана на явном знании, а не на неявном. Дать намек, почему этот тезис так "очевиден" всем в книге.

2. современная медицина почти остановила эволюцию

3. Ускорение эволюции требует жестоких мер.

4. Люди на них не пойдут.

Происходит недолгий спор о научной обоснованности.

Но Александр сворачивает разговор в другое русло: Предположим с теоретико-эволюционной точки зрения все правильно (тезис о ключевой роли личных способностей для усиления военной мощи общества не оспаривается). Но ключевая часть плана Полимегиста состоит в том, чтобы держать народ в неведении относительно истиной природы чумы, иначе он найдет способ от нее избавиться. То есть Полимегист осознает, что убедить людей подвергнуться жестокой селекции добровольно не выйдет. Т. е. Полимегист решает судьбы людей за них. Александр считает, что решение, что делать дальше с угрозой очередного вторжения демонов должно быть результатом общественного компромисса.

Александр и Полимегист в ходе разговора понимают, что их моральные принципы делают их позиции непримиримыми. Полимегист дает понять что уйти ником не даст (когда почуяв жареное самые ушлые захотели улизнуть), раскрывает что примет любые меры, чтобы сохранить секретность своего плана. Напряжение нарастает, почти все кроме Александра и Полимегиста сидят уже скованные страхом. Полимегист видит в глазах Александра сосредоточенную решимость, чувствует что тот уже осознал неизбежность битвы и тянуть больше нельзя.

Полимегист атакует: Асклепия парализует Александра (воспользовавшись тем, что она его лечила во время битвы) и дает Полимегисту мгновение задержки, чтобы он мог убить Александра. Мысль Полимегиста: почему он не атаковал первым? -- те самые принципы, что не дали ему договориться с Полимегистом. Он готовился к обороне, но это было тактически проигрышно, он надеялся на свое преимущество над Полимегистом в бою, но Асклепия стала для него сюрпризом (мало кто знал об их отношениях и насколько бездумна ее любовь).

Тем временем оставшиеся в живых аватары богов впадают в прострацию (по заклинанию подготавливаемому Полимегистом перед советом). Выходы из зала совета блокируются автоматическими системами дворца. Полимегист объявляет оставшимся, что те кого он не сможет убедить следовать своему делу, не выйдут из комнаты живыми.



\section*{Эпоха Чумы}
Ужасающие доказательства правоты Полимегиста.
Эмпирическое наблюдение ускорения эволюции.
Логика введения модификаций в дествие чумы.
Исчезновение столовых приборов (людям стало удобнее заменять их заклинаниями, что раньше практиковалось только как детские понты).

\section*{мысли Ариста}



объединение с вражеским кланом, путем сдачи в плен

воспламенял души огнем своего мистического озарения

Зов мировой души надо было научиться чувствовать, но ведомые и очарованные озарением, полученным от Ходжи, многие достигали этого.


Идеи
\begin{itemize}
	\item Исчезновение "я". Жажда синхронизации со своим "сверх я".
	\item "Праведность" как оружие
	\item Продуманная этика как броня
	\item Спор идеалов индивидуализма и сверх-я. 
\end{itemize}


\section*{Финал}

про зацикленность Полимегиста и нужность diversity.

Бытие не полностью определяет сознание, метафора с накрытием

\end{document}

