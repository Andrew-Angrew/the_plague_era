% !TeX encoding = UTF-8
\documentclass[12pt,a4paper]{article}
%\usepackage{russ}

\usepackage[utf8]{inputenc}
\usepackage[english,russian]{babel}
%\usepackage[left=2cm,right=2cm,top=2cm,bottom=3cm]{geometry}

\usepackage{amsmath,amsfonts,amssymb}
\usepackage{amscd}
\usepackage{amsthm}
%\usepackage{geometry}
%\usepackage{fancyhdr}
%\parindent = 0em %ðàññòîÿíèå îò êðàÿ â êðàñíûõ ñòðîêàõ
\usepackage{color}
\usepackage{cmap}

\usepackage{underscore} %нижние подчеркивания вне формул

%для ссылок
\usepackage{xcolor}
\usepackage{hyperref}
\definecolor{linkcolor}{HTML}{799B03} % цвет ссылок
%\definecolor{urlcolor}{HTML}{799B03} % цвет гиперссылок
\hypersetup{pdfstartview=FitH,  linkcolor=linkcolor, urlcolor=blue
	, colorlinks=true}

\usepackage{ulem} %зачеркивание via \sout{}

% чтобы itemize работал без лишних пробелов
\usepackage{enumitem}
\setlist{nolistsep,
	% itemsep=0.3cm,
	parsep=4pt}

\usepackage{color}
\newcommand{\tr}[1]{\textcolor{red}{#1}}

\newcommand{\todo}[1]{\marginpar{\scriptsize \tr{#1}}}

\newcommand{\sep}{
	\begin{center}
		\line(1,0){300}
	\end{center}
}

\begin{document}


\section*{Предисловие}

События, о которых я хочу рассказать, не имеют удовлетворительного описания. Дело в том, что их участники очень сильно отличаются от людей. Трудно даже провести параллели между их органами чувств и нашими. Возможно, было бы правильнее изображать их, как призраков-колдунов, живущих в атмосфере газового гиганта. Или как морских существ с жидкими телами, способными  смешиваться с другими существами и предметами в одном объеме, но при этом четко осознающими и контролирующими границы собственного тела. Это образы лучше описывали бы физическую суть происходящего. Но чтобы передать то, что мне в этой истории кажется ценным, я буду рисовать своих героев людьми, используя в качестве красок узнаваемые образы из человеческой культуры. Иногда этот антропоморфный образ будет ломаться, несмотря на намеренную неподробность описаний. Надеюсь, это вступление позволит сохранить в такие моменты ощущение осмысленности происходящего.

\section*{Встреча}

Одинокий путник пришел из пустыни к окраине города. По нынешним временам он был необычайно стар. Он не был немощным, возраст только придавал достоинство его облику. Только вот уже много поколений никому не удавалось прожить столько лет. Всех рано или поздно убивала чума. Чумой болели все. Каждая жизнь рано или поздно превращалась в изматывающую и безнадежную борьбу с этой болезнью.

Тело путника было по-архаичному мощным и грузным. Такое сложение можно увидеть только у статуй времен золотого века или в самых давних воспоминаниях наследственной памяти. Его движения казались неестественно точными и равномерными, будто подчиненными своеобразной эстетике, имеющей мало общего с природной грацией.

Сквозь кольцо трущоб, путник добрался до центра города -- перенаселенных античных руин. Повсюду в грязи и пренебрежении были разбросаны следы былого великолепия. Роскошные древние дома облеплены нелепыми пристройками, заселенными множеством семей.

Путник проехал мимо шкуры демона, установленной на центральной площади. Шкура была частично невидима для всех чувств, как черно-белая дыра в цветном мире. В месте прикосновения к ней кожа как-будто немела. Другими словами шкура была \textbf{\textit{мертва}}. Более удивительно было то, что она абсолютно не поддавалась всеразлагающему воздействию \textbf{\textit{жизни}}. И потому вместе с другими шкурами была древнейшей в мире вещью, оставшейся еще со времен первого апокалипсиса.

Доехав до центрального храма-дворца путник представился Авиценной, великим странствующим целителем. Он заявил, что умеет лечить чуму, и запросил аудиенции у самого пророка Ходжи\footnote{Ходжа -- наследственный титул потомков исламских миссионеров суфийского течения, предполагающего, что все души - части единого целого.}. Ссылаясь на обычаи родины, Авиценна отказался открыть свой разум и показать искренность своих намерений. Это было откровенно враждебным действием. Но все же небольшая демонстрация \textbf{\textit{силы}} открыла ему двери дворца, для оказания соответствующего приема. Привилегированность могущественных входит во все обычаи. Через некоторое время Ходжа решил принять гостя, несмотря на всю подозрительность его поведения.

Ходжа был не был юн, но его возраст казалось не соответствовал его славе мудреца и уважению которое оказывали ему его последователи. Он выделялся худощавостью и был физически неразвит, а весил наверное чуть ли не вдвое меньше необычайно мускулистого Авиценны.

Начало аудиенции прошло в тишине, -- увидев больных, Авиценна без лишних слов занялся ими. Взрослых чума выматывает постепенно, а к детям приходит периодически в виде острых приступов. Представлены были все возраста и степени тяжести. Казалось, Авиценна решил осмотреть всех от тяжелых к легким. Но когда он добрался до последних пациентов, стоны детей, осмотренных первыми, начали стихать. Все были исцелены. Вот так просто, без заметных манипуляций и инструментов.

-- Я понимаю, что любое предложение высказанное на словах вызывает недоверие. -- сказал Авиценна -- Надеюсь отпечатка моей мысли, хватит в качестве доказательство моей искренности?
С легкой улыбкой Ходжа кивнул.

-- И еще я бы хотел продолжить разговор в максимально узком кругу.

-- Вы же понимаете, что учитывая {\it наши} обычаи, ничего не получится сдержать в секрете.

-- Да, но мне бы хотелось хоть как-то сдержать распространение информации о том, что я собираюсь сообщить. Это помогло бы в деле, которое я намерен предложить.

По знаку Ходжи все, кроме нескольких ближайших советников, покинули зал. Очевидно он был абсолютно уверен в своих силах, особенно в своем доме, и не испытывал страха перед подзрительным и могущественным чужаком. Авиценна погрузился в изготовление оттиска. Мысль постепенно застывала в предоставленном для нее камне, формируя образ столь многогранный и сложный, что подделка его считалась невозможной. При прочтении камня в сознании разом вспыхивали впечатления, которые можно было примерно передать следующими словами:

\begin{quotation}
	\textit{Моя цель -- спасение человечества от гибели. Но этот проект в отличие от моей жизни далек от завершения. Многие детали ради успеха дела я вынужден держать в секрете, поэтому не могу никому открыть свой разум. Я хочу, чтобы Ходжа стал моим приемником. Я уже пытался с более слабыми кандидатами, но они не справились. Поэтому, Ходжа, твое согласие очень важно.}
\end{quotation}

-- Почему вы решили, что именно я вам нужен?

-- Трудно найти равного вам по силе и мастерству. А подвергать сомнению ваши моральные достоинства, особенно в этой стране, кажется кощунством. -- сказал Авиценна с легкой усмешкой -- Также я изучил ваши трактаты по этике и теории распределенных систем и оценил ваш блестящий ум. Все эти качества важны для продолжения моего дела. Уверен подобное сочетание всех добродетелей в одном лице уникально.

В ходе дальнейшей беседы Ходжа понял, что больше информации из Авиценны не вытянуть и вскоре дал согласие. Всего через день он был готов оставить свой пост. Весть об этом еще не успела распространиться на всю общину. Поэтому, как и хотел Авиценна, они вдвоем смогли отбыть инкогнито.

Когда они углубились в пустыню Авиценна сказал:

-- Теперь нам нужно телепортираться. Позволь наложить на тебя заклинание.

Исполнение просьбы поставило бы Ходжу в уязвимое положение. После короткой паузы он спросил:

-- Почему бы вам просто не открыть мне свой разум? Я ведь все равно должен узнать ваши секреты и теперь я изолирован.

-- Горький опыт подсказывает, что прежде, чем это сделать, нужна теоретическая и моральная подготовка. Без телепортации мы не сможем попасть в мой дом, где собраны необходимые инструменты. Кроме того процес обучения предполагает множество ситуаций, когда ты окажешься в моей власти. Прошу, доверься мне.

-- Почему вы предпочитаете логику и ультиматумы взаимопониманию и компромиссам? - с легким вздохом смирился Ходжа.
Оставив вопрос без ответа Авиценна сотворил заклинание.



\section*{Первый апокалипсис}

Огненные линии в небе завершились далеким грохотом. Никто не смог истолковать этот знак, и мы отправились по растаявшему в воздухе следу. На самой границе наших кочевий мы нашли место падения. Глубокая рана в теле земли уже затягивалась: мертвый пепел превращался в почву, которая по краям кратера успела дать побеги. Но несмотря на старания жизни вернуть свое, никто не мог припомнить такого количества мертвого вещества, на сколько бы поколений вглубь памяти не удавалось заглянуть. Наблюдать мертвое было странно и интересно. Даже сосредоточившись можно ощутить лишь очень небогатую гамму ощущений. Как будто трогаешь что-то обмороженными руками. Привычные манипуляции с ним не удаются. Как будто толкаешь что-нибудь, но внезапно не встречаешь сопротивления.

Вещество в центре кратера становилось все более прозрачным и доступным восприятию по мере того, как жизнь просачивалась внутрь. И вот мы заметили в нем смутное движение гигантских, черезвычайно плотных фигур. Их вид казался не просто странным. Он был невозможным. Одни чувства вместе с землей содрогались от мощи их движений. Другие видели на их месте только идеальную, пугающую пустоту. Все что приближалось к ним тоже постепенно начинало ускользать от наших чувств. Попытки прощупать их вызвали длинные струи ярчайшего пламени. Но никого из нас не задело -- отпор гигантов был разрушительным, но неуклюжим. Издалека мы попробовали спровоцировать их еще и вскоре пришли к выводу, что они нас не видят. Они не замечали никакой активности вокруг них, пока та не выливалась в грубое физическое воздействие. Но как только рядом что-то перемещались, они испепеляли это, игнорируя даже самые очевидные следы, ведущие к источнику соответствующих манипуляций.

Мертвые существа не казались особенно опасными, но производили впечатление чего-то пугающе неправильного. Уходя на стоянку, мы решили, что это дурной знак.

Когда мы в следующий раз пришли проведать кратер, мы ожидали что все мертвое вещество будет уже поглощено жизнью. Но его стало только больше. Потом мы увидели самое страшное. Вещество кратера ощущалось как мертвое, но внутри кипела деятельность, как будто оно было живым. Мы заметили, как смерть борется с окружающей его жизнью и постепенно разъедает ее, как кислота. Мы долго пытались что-нибудь с этим сделать. Но раз за разом получали отпор и в конце концов привлекли внимание демонов (так мы окрестили мертвых гигантов). Больше всего пугало то, что все мертвое реагировало на нас как единое целое. Как одно невозможно большое и могучее существо, способное легко раздавить всех нас. Как и демоны оно не отличалось умом, но он был ему и не нужен, чтобы оттеснять нас.

Вскоре нам пришлось бежать с наших земель.

\sep

Мою семью миновали лишения бесконечного похода. Но все равно мое счастье и детство кончились рано. Сначала меня стала преследовать беспричинная тревога. Желая понять ее причину, я прислушался к своим чувствам. И тогда из глубин сознания поднялся холодный неумолимый ужас... Дальше я помню только, как уже очутился в объятиях у родителей, они молча успокаивали меня. А потом я начал вспоминать... чуждое, непонятное, мертвое зло, что пожрало каждый клочок земли, пройденной за долгие годы... изнуряющее бегство от края черной пропасти размером в пол-мира... постоянное расчеловечивающее ожесточение борьбы за жизнь с другими беженцами...

Ко дню инициации я помнил уже достаточно, чтобы четко осознать, насколько случайно получилось так, что я живу. Что я нисколько не лучше многих тех, кто сгинул в хаосе похода вместо моих предков. Опыт многих поколений говорил мне, что я вероятно рано умру, чтобы мои близкие могли умереть чуть позже.

Обреченность довлела над всеми. Кто-то добровольно уходил из жизни. Кто-то с мрачным упорством ходил на край мертвого моря, пытаясь как-то на него повлиять. И однажды случилось невероятное. Одна маленькая, почти не ощутимая частичка мертвого вещества откликнулась на манипуляцию. С помощью нее удалось захватить еще и еще. И вот в распоряжении героя оказался уже приличный объем. Секрет было трудно объяснить -- только долгая ментальная синхронизация помогала. Несмотря на это, он быстро разлетелся по племенам беженцев. В течение десятка лет мы достигли мастерства, позволявшего мечтать о том, тобы владеть мертвым морем, как мы когда-то владели всем живым миром.

Но мертвое море продолжало наступать. Наконец, мы уже достаточно овладели мастерством демонической жизни, чтобы обмануть море и перемещаться в нем. Некоторые племена отважились снарядить экспедиции вглубь моря. Когда мы полностью погрузились в море демоническое мастерство стало работать гораздо эффективнее. Как будто все вещи стали очень легкими и прозрачными, а силы трения были отменены. Демоническое мастерство всегда казалось далеко отстающим аналогом привычного мастерства манипуляции жизнью. Но внутри моря стали открываться возможности немыслимые для мастерства жизни.

Однажды мы решили, что можем сразиться с морем его же силами. Мы прокрались в черную глубину так далеко, как только смогли, чтобы демонические способности стали максимально эффективны. Наконец, самый способный из нас осуществил задуманное. Он как бы закричал. 
Но этот крик вместо того, чтобы затухать только набирал силу в среде мертвого моря. Волна распространяясь все дальше набирала колоссальную мощь. Потом мы ощутили, как где-то на расстоянии, недоступном для обычного восприятия, обрушились горы...

В ответ на нашу наглость мертвое море вокруг нас умерло, лишилось своей демонической жизни. Превратилось в обычную мертвую непрозрачную массу. Было очень больно. Как оказалось мы бессознательно включили в наши тела частички демонической материи. А теперь они внутри нас обратились пепел. Чем лучше владение демоническим мастерством, тем больше был урон. Некоторые из нас умерли. Мы были ранены и пойманы в мертвом веществе, как в зыбучем песке. Поскольку я пострадал меньше всех, было решено телепортировать меня домой просить подмогу. Но как только я оказался дома, гиганский взрыв поглотил место нашей вылазки вместе с огромным умершим куском моря.

Из нашей группы выжил только я, но это была победа! Началось отвоевание земли. Море по частям отмирало, пытаясь защититься от наших атак. Затем кому-то удалось отравить его и убить полностью. Остатки демонов еще долго бродили по оживающему миру, постепенно погибая от рук все новых героев.


\end{document}

