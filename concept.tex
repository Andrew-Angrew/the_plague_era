% !TeX encoding = UTF-8
\documentclass[12pt,a4paper]{article}
%\usepackage{russ}

\usepackage[utf8]{inputenc}
\usepackage[english,russian]{babel}
\usepackage[left=3cm,right=3cm,top=3cm,bottom=3cm]{geometry}

\usepackage{amsmath,amsfonts,amssymb}
\usepackage{amscd}
\usepackage{amsthm}
%\usepackage{geometry}
%\usepackage{fancyhdr}
%\parindent = 0em %ðàññòîÿíèå îò êðàÿ â êðàñíûõ ñòðîêàõ
\usepackage{color}
\usepackage{cmap}

\usepackage{underscore} %нижние подчеркивания вне формул

%для ссылок
\usepackage{xcolor}
\usepackage{hyperref}
\definecolor{linkcolor}{HTML}{799B03} % цвет ссылок
%\definecolor{urlcolor}{HTML}{799B03} % цвет гиперссылок
\hypersetup{pdfstartview=FitH,  linkcolor=linkcolor, urlcolor=blue
	, colorlinks=true}

\usepackage{ulem} %зачеркивание via \sout{}

% чтобы itemize работал без лишних пробелов
\usepackage{enumitem}
\setlist{nolistsep,
	% itemsep=0.3cm,
	parsep=4pt}

\usepackage{color}
\newcommand{\tr}[1]{\textcolor{red}{#1}}

\newcommand{\todo}[1]{\marginpar{\scriptsize \tr{#1}}}

\newcommand{\sep}{
	\begin{center}
		\line(1,0){300}
	\end{center}
}

\begin{document}

\pagestyle{empty}

\subsection*{Особенности мира:}

Почти вся материя мира является \textit{живой} и есть магия, которая называется \textit{мастерством}. Мастерство позволяет более или менее произвольные манипуляции с живым веществом. В той или иной степени мастерством обладают все вплоть до растений. Люди сильно отличаются по мастерству, сильнейшие в боевом отношении равносильны армии обычных людей.
Существуют несколько форм телепатии, но чем более сложная информация передается, тем сильнее искажаются разумы обоих участников. Т. е. это не передача, а синхронизация с плохо контролируемыми побочными эффектами.
Также люди частично наследуют память и навыки родителей.

\subsection*{История мира}
Первый апокалипсис. Еще в незапамятную, первобытную эпоху случился первый апокалипсис. С неба упали демоны -- мертвые существа, превращающие все вещество в мертвое. Вся жизнь тогда чуть не погибла.

Золотой век. Через долгое время произошла технологическая революция: люди научились делать артефакты -- предметы, позволяющие применять заклинания, вложенные в них создателем. Применение артефактов позволило ''растиражировать'' таланты исключительных мастеров на все человечество. Произошел огромный рост населения, благосостояния и продолжительности жизни. Угнетение из-за материальных благ ушло в прошлое, сложилось анархическое, гуманистическое общество, сфокусированное на развитии уникальных индивидуальных способностей. Тесные телепатические контакты не приветствуются.

Второй апокалипсис. Золотой век кончился вторым апокалипсисом. Человечество выжило, но опять оказалось на грани исчезновения, несмотря на то, что его мощь со времен первого апокалипсиса невероятно возросла. Большинство людей оказалось абсолютно бесполезно и не участвовало в битве, а боеспособная верхушка общества наоборот полегла почти в полном составе. Немногие выжившие магистры вскоре погибли в междоусобице и большое количество самых важных технологий оказалось недоступно для использования. Вдобавок началась эпидемия \textit{чумы}. 

Эпоха чумы. Чума -- медленно и мучительно убивающая болезнь. Высокое мастерство позволяет сопротивляться ей ценой все возрастающих усилий, но в конечном итоге умирают все. Поэтому у человечества перестало хватать сил и времени на поддержание старых социальных институтов. Люди больше стали полагаться на телепатию для передачи навыков между быстро сменяющимися поколениями, обеспечения доверия в условиях беззакония и быстрого согласования интересов в коллективах (с помощью телепатии невозможно врать). Спустя некоторое количество поколений люди организуются в \textit{единодушия} -- группы, внутри которых постоянно поддерживается телепатический \textit{консенсус} -- общая для всех система целей и эмоциональных оценок. Единодушия редко превышают 200 человек из-за того невозможности поддерживать консенсус.

\subsection*{Сюжет}

Старый целитель Авиценна (на самом деле Полимегист -- выживший магистр золотого века) уговаривает Ходжу стать его учеником и продолжить очень важное дело, суть которого пока не ясна. Ходжа -- исключительно могущественный и умный маг, а также пророк нового религиозного движения. Полимегист считает, что только он способен продолжить его дело. В ходе обучения раскрывается информация об устройстве и истории мира. А в конце Полимегист, чтобы объяснить свои планы и мотивы, открывает свой разум для телепатической синхронизации. Герои видят воспоминания и мировоззрение друг друга.

Твист 1: Чуму наколдовал Полимегист, чтобы ускорить эволюцию человеческих способностей для подготовки к следующему апокалипсису.

Твист 2: Религиозное учение Ходжи о растворении в Боге оказывается не только мистическим откровением, но и вполне рациональным проектом объединения человечества в единый сверхразум. Сама природа людей изменилась за многие поколения жизни в единодушиях и подобное растворение является их естественным устремлением.

Растворение вызывает у Полимегиста экзистенциальный ужас и кажется равносильным исчезновению человечества. Чтобы помешать растворению он убивает Ходжу и прекращает чуму. Судьба человечества неясна.

Твист 3 (в книге будут только намеки): Герои книги -- это эволюционирующие протоколы, управляющие наномашинами в гигантском супе из наномашин, в который превращена их планета в ходе неудачной попытки колонизации рассой прародителей. Апокалипсисы -- попытки прародителей исправить проблему.

\subsection*{Отдельные идеи}
\begin{itemize}
\item Дарвиновская эволюция и эволюция по Ламарку могут сосуществовать. Дарвиновский механизм требует обновления генофонда т. е. смертей.
\item Бытие не определяет сознание, а только ограничивает ему возможности, поэтому нельзя все сделать как было (изменившиеся люди и без чумы стремятся к растворению).
\end{itemize}

\end{document}

