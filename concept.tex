% !TeX encoding = UTF-8
\documentclass[12pt,a4paper]{article}
%\usepackage{russ}

\usepackage[utf8]{inputenc}
\usepackage[english,russian]{babel}
\usepackage[left=2cm,right=2cm,top=2cm,bottom=3cm]{geometry}

\usepackage{amsmath,amsfonts,amssymb}
\usepackage{amscd}
\usepackage{amsthm}
%\usepackage{geometry}
%\usepackage{fancyhdr}
%\parindent = 0em %ðàññòîÿíèå îò êðàÿ â êðàñíûõ ñòðîêàõ
\usepackage{color}
\usepackage{cmap}

\usepackage{underscore} %нижние подчеркивания вне формул

%для ссылок
\usepackage{xcolor}
\usepackage{hyperref}
\definecolor{linkcolor}{HTML}{799B03} % цвет ссылок
%\definecolor{urlcolor}{HTML}{799B03} % цвет гиперссылок
\hypersetup{pdfstartview=FitH,  linkcolor=linkcolor, urlcolor=blue
	, colorlinks=true}

\usepackage{ulem} %зачеркивание via \sout{}

% чтобы itemize работал без лишних пробелов
\usepackage{enumitem}
\setlist{nolistsep,
	% itemsep=0.3cm,
	parsep=4pt}

\usepackage{color}
\newcommand{\tr}[1]{\textcolor{red}{#1}}

\newcommand{\todo}[1]{\marginpar{\scriptsize \tr{#1}}}

\newcommand{\sep}{
	\begin{center}
		\line(1,0){300}
	\end{center}
}

\begin{document}

\thispagestyle{empty}

\subsection*{Особенности мира:}

Почти вся материя мира является \textit{живой} и как-бы магия, которая называется \textit{мастерством}. Мастерство позволяет более или менее произвольные манипуляции с живым веществом. В той или иной степени мастерством обладают все вплоть до растений. Люди сильно отличаются по мастерству, сильнейшие в боевом отношении равносильны армии обычных людей.
Существуют несколько форм телепатии, но чем более сложная информация передается, тем сильнее искажаются разумы обоих участников. Т. е. это не передача, а синхронизация с плохо контролируемыми побочными эффектами.
Также люди частично наследуют память и навыки родителей.


\subsection*{История мира}
Первый апокалипсис
Артефакты
Эпоха просвещения
Второй апокалипсис
Смерть магистров
Эпоха чумы

\subsection*{Сюжет}

Взятие Ходжи в ученики
Синхронизация

Твист 1: Полимегист наслал чуму, чтобы ускорить эволюцию человеческих способностей для подготовки к следующему апокалипсису.

Твист 2: Религиозное учение Ариста о растворении в Боге оказывается не только мистическим откровением, но и вполне рациональным проектом объединения человечества в единый сверхразум. Сама природа людей изменилась за многие поколения жизни в единодушиях и подобное растворение является их естественным устремлением.

Растворение вызывает у Полимегиста экзистенциальный ужас и кажется равносильным исчезновению человечества. Чтобы помешать растворению он убивает Ариста и прекращает чуму. Судьба человечества неясна.

Твист 3 (в книге будут только намеки): Герои книги -- это эволюционирующие протоколы, управляющие наномашинами в гигантском супе из наномашин, в который превращена их планета в ходе неудачной попытки колонизации рассой прародителей. Апокалипсисы -- попытки прародителей исправить проблему.

\subsection*{Отдельные идеи}
\begin{itemize}
\item Дарвиновская эволюция и эволюция по Ламарку могут сосуществовать. Дарвиновский механизм требует обновления генофонда т. е. смертей.
\item Бытие не определяет сознание, а только ограничивает возможности, поэтому нельзя все сделать как было.
\end{itemize}

\end{document}

